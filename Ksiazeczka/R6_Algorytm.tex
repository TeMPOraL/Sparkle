%TODO dopisać coś wiecej dlaczego mniejsze komórki nie są potrzebne.
% skala  budynku to że nie interesue nas jak dokładnie będzie się palić każde 10cm^3 ale 
% jak ogien będzie się przemieszczał. rozmiar człowieka z perełe
\chapter{Algorytm}
\label{cha:Algorytm}
Rozdział przedstawia propozycję algorytmu symulacji rozprzestrzeniania się ognia i dymu podczas pożaru.
Przedstawiony poniżej model jest niehogenicznym automatem komórkowym i jako taki spełnia postulat niehomogeniczności.
W pierwszej części rozdziału przedstawiono wartości parametrów tworzących automat komórkowy. Kolejne podrozdziały 
zawierają szczegółowy opis kluczowych funkcji współtworzących funkcję przejścia.
% Plan rozdziału
% 1. Jaki typ automatu 3D. wielkość - określana przez usera
% 2. Kształt komórek i wielkość komórek
% 3. Sąsiedztwo - nieregularne, inni sąsiedzi przy krawedziach
% 4. Zbiór stanow
% 5. Funkcja przejscia
% 6. W kolejnych podrozdziałach funkcje będące czynnikami funkcji przejścia - przewodnictwo, konwekcja, dym
% 7. Rodzaje komórek. problem wąskich drzwi i jego rozwiązanie
\section {Model automatu}
Zgodnie ze wzorem \ref{def_automatu} będącym istotą przytoczonej w rozdziale \ref{cha:Automaty komórkowe} definicji automatu komórkowego według Weimara jednym z kluczowych elementów jest określenie siatki, czyli powierzchni automatu. W modelu symulaci pożaru w budynku
ze względu na trójwymiarowość zjawiska oraz istotę jego rzeczywistego odtworzenia (możliwość wykorzystania wyników w celu
opracowania modelu ewakuaci osób, badanie przyczyn katastrofy i drogi rozchozenia ognia) nabardziej naturalnym typem automatu 
jest automat \textsl {trójwymiarowy}. 
Rozmiar automatu jest wielkością zmienną, definiowaną przez użytkownika systemu. Pozwala to na odpowiedni dobór ilości komórek w zależności od wielkości rozpatrywanego budynku. Ze względu na fakt, że istotą przedstawionego modelu jest symulacja pożaru wewnątrz budynku 
Zastosowany twójwymiarowy automat składa się z szcześciennych komórek o wymiarach $0.5m x 0.5m x 0.5m$. Wielkość komórek została wybrana empirycznie. 
Odpowiedni dobór wielkości komórek automatu ma kluczowy wpływ na jego działanie. Zbyt mała ilość komórek może doprowadzić do utraty
dokładności algorytmu oraz ukazać zniekształcony obraz działania modelu. Zbyt duża ilość elementów powoduje spadek wydajności algorytmu, a w komputerowej realizacji algorytmu oznacza zwiększone zapotrzebowanie na pamięć i moc procesora.
Wybrany na podstawie doświadczeń rozmiar komórki jest najlepszym
kompromisem między między czasem działania a dokładnością modelu.