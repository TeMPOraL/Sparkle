\chapter{Teoria}
\label{cha:Teoria}
Kluczowym elementem, niezbędnym do prawidłowego zamodelowania pożaru
jest zrozumienie czym jest ogien, poznanie zjawisk jakim podlega.Niniejszy rozdział zawiera
krótki wstęp teoretyczny, przedstawiający zjawiska fizyczne niezbędne do zrozumienia istoty 
pożaru i prawidłowego jego zamodelowania.
\section {Czym jest ogień}
\section {Proces spalania}
\section {Propagacja ciepła}
Jak zostało wspomniane w rozdziale \ref{Proces spalania} jednym z czynników niezbędnych
do podtrzymania ognia jest cieło. Ciepło podczas pożaru jest propagowane na trzy różne sposoby:
\begin {itemize}
\item Przewodnictwo
\item Konwekcja
\item Radiacja
\end {itemize}


\subsection {Przewodnictwo}
 Przewodnictwo cieplne jest procesem, który polega  na wymianie ciepła 
pomiędzy nierównomiernie ogrzanymi ciałami będącymi w kontakcie. Zachodzi ono we wszystkich stanach skupienia: ciałach stałych, cieczach i gazach, jednak sposób i skala tego zjawiska jest bardzo zróżnicowana. Najczęściej mówimy o przewodnictwie w ciałach stałych.
W cieczach i gazach występuje ono niezmiernie rzadko i polega na zderzeniach cząsteczek podlegających
niezorganizowanym, przypadkowym ruchom i ich dyfuzji.
W ciałach stałych przenoszenie ciepła odbywa się  na dwa sposoby:
\begin{itemize}
\item dzięki drganiom atomów
\item poprzez ruch elektronów
\end {itemize}
Celem omawianego przewodnictwa jest 
osiągnięcie równowagi cieplnej. Podczas przewodnictwa ciepło jest zawsze przenoszone od
ciała o większej temperaturze do ciała o niższej. Zgodnie z zasadą zachowania energi, głoszącą że w układzie 
izolowanym suma wszystkich energii jest stała, ilość energii uzyskanej przez ciało chłodniejsze jest równa
ilości energii oddanej przez cieplejszy obiekt. Energia przenoszona jest wraz z ruchem cząsteczek wewnętrznych.
Nie wszystkie ciała przewodzą ciepło w takim sam sposób.
Zależność między ilością ciepła przewodzonego przez ciało, a jego zmianą temperatury najlepiej opisuje prawo Fouriera.
Przyjmuje ono następującą postać:
\begin{equation}
 q(r,t)=-k*grad T
 \label{eqn:fourier}
\end {equation}
gdzie:
k - współczynnik przewodzenia ciepła $[W / (m*K)]$
T - temperatura $[K]$
q - natężenie strumienia ciepła  $[W/(m^2)]$
Prawo Fouriera oznacza, że gęstość strumienia ciepła przekazywana w jednostce czasu przez jednostkową powierzchnię 
jest proporcjonalna do gradientu temperatury. Minus we wzorze wynika ze wspomianego wyżej kierunku przepływu ciepła:
od ciała cieplejszego do zimniejszego. Strumień ciepła jest mierzony w kierunku zgodnym z jego przepływem, zatem przyrost
temperatury będzie miał wartość ujemną.


Do dobrych przewodków należą przede wszystkim:
\begin {itemize}
\item metale - do najlepszych należą srebro, miedź, złoto, aluminium
\end {itemize}
Źle przewodzą ciepło:
\begin {itemize}
\item drewno
\item papier
\item ciecze
\item gazy
\end {itemize}
Zła przewodność cieczy i gazów wynika z istoty procesu przewodnictwa w tych stanach skupienia. Za wysoką wartość 
współczynnika przewodnictwa odpowiada ruch elektronów. Dlatego też, we wszystkich dialektrykach wartość ta
przyjmuje wartości z przedziału $[0,001-3][W/(m*K)]$, podczas gdy w metalach może siegać ona nawet $ 400 [W/m*k]$

Na uwagę zasługuje też fakt, że przewodność metali maleje wraz ze wzrostem ich temperatury.
Tabela \ref {przewodnictwa} zawiera współczynniki przewodnictwa przykładowych materiałów, które zostały wykorzystane przy
testowaniu algorytmu symulacji.
%http://tabelechemiczne.chemicalforum.eu/przewodnictwo_ciala.html
%Dodać do bibliografii tablice fizyczne z których to wzięte
\begin{table}

\begin {center}
\begin{tabular} {|l | c | c|}
\hline
Materiał & Temp. $[C]$ & Wspł. przewodnictwa $[W/(m*K)]$ \\ \hline
Beton & 20 & 0.84-1.3  \\ \hline
Drewno & - & 0.1-0.17  \\ \hline
Szklo crown & 20 & 0.22-0.29  \\ \hline
Azbest & 20 & 0.16-0.37 \\ \hline
Guma wulkanizowana & 20 & 0.22-0.29 \\ \hline
Miedź & 20 & 400 \\ \hline
Stal & 20 & 10 \\ \hline
Ołów & 20 & 30 \\ \hline
\hline
\end {tabular}
\caption{Współczynniki przewodnictwa materiałów}
\end{center}
\end {table}
\subsection{Konwekcja}

\subsection {Radiacja}
Radiacja, czyli inaczej promieniowanie jest to sposób rozchodzenia ciepła w postaci fal elektromagnetycznych. 
Najważniejszym aspektem przewodnictwa ciepła przez promieniowanie jest możliwość wymiany ciepła między ciałami
nie stykającymi się.
W bardzo niskich temperaturach ilość przekazywanego przy pomocy radiacji ciepła jest tak mała, że zjawisko to może być pomijane.
Wzrost znaczenia promieniowania następuje wraz ze wzrostem temperatury ciał wymieniających ciepło.
Przyjmuje się, że radiacja zachodzi dla ciał o temperaturach wyższych od $0 ^\circ$ Kelvin.
Promieniowanie jest rodzajem wymiany energii, która nie wymaga żandego nośnika. Każde ciało emituje fale.
W normalnych warunkach większość promieniowania zachodzi przy udziale fal podczerwonych. Należy jednak pamiętać
że w radiacji mogą brać udział także fale świetlne czy ultrafioletowe. Poza emisją promieniowania każde ciało reaguje także
na fale wysyłane przez innych. Dla każdego ciała jesteśmy w stanie określić wartości trzech współczynników opisujących 
reakcję ciała na wiązkę promieniowania. Należą do nich:
\begin {itemize}
\item Absorpcyjność czyli pochłanialność
\item Refleksyjność czyli odbijalność
\item Przepuszczalność
\end {itemize}
Radiacja następuje we wszystkich kierunkach aż  do momentu zablokowania drogi promieni przez ciało pochłaniające je.
W przypadku jednej powierzchni zamkniętej w innej zakłada się, że wewnętrzna powierzchnia "nie opromieniowuje samej siebie".
Innymi słowy całe promieniowania ciała wewnętrznego przechodzi do powierzchni zewnętrznej. W drugim kierunku następuje tylko częściowe
przejście energii z ciała zewnętrznego do wewnątrz.
Większość ciał stałych o rozmiarach większych od kilku mikrometrów nie przepuszcza promieniowania. Na wspomnianej 
głębokości pod powierzchnią ciała następuje całkowita absorpcja promieniowania cieplnego. Ponadto ciała stałe mogą przepuszczać
fale tylko o określonej długości. Przykładem jest szkło, które przepuszcza jedynie fale świetlne.
Ilość promieniowania emitowanego przez ciała szare można obliczyć ze wzoru \ref{emisyjnosc}
\begin {equation}
Q_{emit}=\sigma*\varepsilon*A_{s}*T_{s}^4
\label {emisyjnosc}
\end {equation}
gdzie
$\varepsilon \in (0,1)$ - emisyjność powierzchni. $\sigma=1$ - dla ciała doskonale czarnego. Określa jak bardzo dane 
ciało jest podobne do ciała doskonale czarnego.
$\sigma = 5.67 * 10^-8 [W/(m^2 * K^4]) $ - stała promieniowania
$A_{s} [m^2]$ - powierzchnia 
$T_{s} [K]$ - temperatura




Rozważając wymianę ciepła przez promieniowanie poza pochłanialnością, refleksyjnością i przepuszczalnością należy wziąć pod uwagę
wzajemną orientację powierzchni oraz. Wszystkie te czynniki wpływają na komplikację promieniowania.

\subsection {Przenikanie ciepła}
%TODO umiescic gdzie indziej. to jest zależnośc temp ciala od dostarczonej energii
\begin {equation}
Q=c*m*\Delta T
\label {ilosc_ciepla}
\end {equation}
gdzie:
\begin {itemize}
\item c - ciepło właściwe $[J/ (kg * K)]$, jego wartość oznacza ile ciepła należy dostarczyć aby ogrzać 1kg substacji o wartość
1K.
\item m - masa ciała $[kg]$
\item T - temperatura  $[K]$
\end {itemize}
