%TODO:
%Wstawić rysunek konwekcji narysowany ze screena z Rosenbajgerowymi strzałkami
%Konwekcja, a grawitacja i prawo Archimedesa
\chapter{Teoria}
\label{cha:Teoria}
Kluczowym elementem, niezbędnym do prawidłowego zamodelowania pożaru
jest zrozumienie czym jest ogień oraz poznanie zjawisk jakim podlega. Niniejszy rozdział zawiera
krótki wstęp teoretyczny, przedstawiający zjawiska fizyczne niezbędne do zrozumienia istoty 
pożaru i prawidłowego jego zamodelowania.
\section {Czym jest ogień}
Ogień nie jest substancją.
Ogień powstaje jako produkt reakcji chemicznej zachodzącej między paliwem i tlenem.
Obserwowalną postać ognia, czyli to co widzimy i nazywamy ogniem tworzy światło powstałe w wyniku ruchu rozgrzanego powietrzna.
Jest to jednak tylko jeden z aspektów tego złożonego procesu.
Elementami koniecznymi do powstania i podtrzymania ognia są:
\begin{itemize}
\item tlen
\item paliwo
\item ciepło
\end{itemize}


Paliwem może być ciecz, ciało stałe lub gaz. Samo w sobie paliwo nie ulega spalaniu. 
Paliwo pod wpływem ciepła pochodzącego np. z zapałki lub otrzymanego od innego nagrzanego ciała ogrzewa się. Po osiągnięciu
odpowiedniej temperatury, paliwo ulega procesowi dekompozycji. Jednym z produktów dekompozycji
są opary. W przypadku jednego z najbardziej popularnych paliw - drewna oprócz oparów w wyniku dekompozycji otrzymujemy węgiel i popiół.
Spalanie drewna przedstawia poniższa reakcja \ref{reakcja_spalania}:
\begin {equation}
CH_2O+O_2+heat ->CO_2 + CO + C + N_2 + H_2O
\label {reakcja_spalania}
\end {equation}
Kiedy opary osiągną odpowiednią temperaturę tzw. temperaturę zapłonu (w przypadku drewna wynosi ona ok. $300^\circ C$) oraz gdy ich stężenie
w powietrzu jest odpowiednie może dojść do zapłonu. Do zapłonu dochodzi w wyniku kontaktu z otwartym ogniem, iskrą lub w wyniku osiągnięcia
przez opary temperatury tzn. samozapłonu. Wynikiem zapłonu jest spalanie oparów. Jak widać głównym substratem reakcji spalania są opary powstające w wyniku
ogrzania paliwa. W przypadku niektórych paliw jest to jedyny reagent. Jednym z przykładów jest benzyna, która w wyniku ogrzania w całości zamienia się w opary ulegające spalaniu. W przypadku drewna, poza oparami spalaniu ulega także węgiel. Jest to jednak reakcja bardzo powolna.
Na szczególną uwagę zasługuje fakt wytwarzania energii cieplnej w procesie spalania, co powoduje samoistne podtrzymanie ognia. Płomień będący
wizualną postacią spalania ogrzewa sąsiadujące cząsteczki paliwa, dzięki czemu nie gaśnie.

Bardzo istotnym reagentem w procesie spalania jest tlen. Atomy gazów, oparów powstałych w wyniku podgrzania paliwa w wyniku zapłonu łączą się z tlenem.
Aby mogło dojść do reakcji spalania bardzo ważne jest zachowanie odpowiednich proporcji między substratami reakcji. Lower Explosive Limit (LEL) określa 
minimalne stężenie oparów w powietrzu, konieczne aby mogło dojść do zapłonu. Odpowiednio, Upper Explosive Limit (UEL) oznacza maksymalne stężenie oparów, powyżej
którego nie dojdzie do zapłonu. Przykładowo: dla tlenku węgla $LEL=12$, natomiast  $UEL=75$, co oznacza w powietrzu musi być między $12\%-75\%$ aby mogło dojść
do jego zapalenia. $LEL$ oraz $UEL$ określają także pośrednio wymaganą ilość tlenu. Dla większości paliw ilość tlenu wymaganego do zapłonu wynosi ok. $15\%$. 

\section {Propagacja ciepła}
Jak zostało wspomniane w rozdziale \ref{Proces spalania} jednym z czynników niezbędnych
do podtrzymania ognia jest ciepło. Ciepło podczas pożaru jest propagowane na trzy różne sposoby:
\begin {itemize}
\item Przewodnictwo
\item Konwekcja
\item Radiacja
\end {itemize}

\subsection {Przewodnictwo}
\label{Przewodnictwo}
 Przewodnictwo cieplne jest procesem, który polega  na wymianie ciepła 
pomiędzy nierównomiernie ogrzanymi ciałami będącymi w kontakcie. Zachodzi ono we wszystkich stanach skupienia: ciałach stałych, cieczach i gazach, jednak sposób i skala tego zjawiska jest bardzo zróżnicowana. Najczęściej mówimy o przewodnictwie w ciałach stałych.
W cieczach i gazach występuje ono niezmiernie rzadko i polega na zderzeniach cząsteczek podlegających
niezorganizowanym, przypadkowym ruchom i ich dyfuzji.
W ciałach stałych przenoszenie ciepła odbywa się  na dwa sposoby:
\begin{itemize}
\item dzięki drganiom atomów
\item poprzez ruch elektronów
\end {itemize}
Celem omawianego przewodnictwa jest 
osiągnięcie równowagi cieplnej. Podczas przewodnictwa ciepło jest zawsze przenoszone od
ciała o większej temperaturze do ciała o niższej. Zgodnie z zasadą zachowania energii, głoszącą że w układzie 
izolowanym suma wszystkich energii jest stała, ilość energii uzyskanej przez ciało chłodniejsze jest równa
ilości energii oddanej przez cieplejszy obiekt. Energia przenoszona jest wraz z ruchem cząsteczek wewnętrznych.
Nie wszystkie ciała przewodzą ciepło w takim sam sposób.
Zależność między ilością ciepła przewodzonego przez ciało, a jego zmianą temperatury najlepiej opisuje prawo Fouriera.
Przyjmuje ono następującą postać:
\begin{equation}
 q(r,t)=-k*grad T
 \label{eqn:fourier}
\end {equation}
gdzie:
k - współczynnik przewodzenia ciepła $[W / (m*K)]$
T - temperatura $[K]$
q - natężenie strumienia ciepła  $[W/(m^2)]$
Prawo Fouriera oznacza, że gęstość strumienia ciepła przekazywana w jednostce czasu przez jednostkową powierzchnię 
jest proporcjonalna do gradientu temperatury. Minus we wzorze wynika ze wspominanego wyżej kierunku przepływu ciepła:
od ciała cieplejszego do zimniejszego. Strumień ciepła jest mierzony w kierunku zgodnym z jego przepływem, zatem przyrost
temperatury będzie miał wartość ujemną.


Do dobrych przewodników należą przede wszystkim:
\begin {itemize}
\item metale - do najlepszych należą srebro, miedź, złoto, aluminium
\end {itemize}
Źle przewodzą ciepło:
\begin {itemize}
\item drewno
\item papier
\item ciecze
\item gazy
\end {itemize}
Zła przewodność cieczy i gazów wynika z istoty procesu przewodnictwa w tych stanach skupienia. Za wysoką wartość 
współczynnika przewodnictwa odpowiada ruch elektronów. Dlatego też, we wszystkich dialektrykach wartość ta
przyjmuje wartości z przedziału $[0,001-3][W/(m*K)]$, podczas gdy w metalach może sięgać ona nawet $ 400 [W/m*k]$

Na uwagę zasługuje też fakt, że przewodność metali maleje wraz ze wzrostem ich temperatury.
Tabela \ref {przewodnictwa} zawiera współczynniki przewodnictwa przykładowych materiałów, które zostały wykorzystane przy
testowaniu algorytmu symulacji.
%http://tabelechemiczne.chemicalforum.eu/przewodnictwo_ciala.html
%Dodać do bibliografii tablice fizyczne z których to wzięte
\begin{table}
\begin {center}
\begin{tabular} {|l | c | c|}
\hline
Materiał & Temp. $[C]$ & Wspł. przewodnictwa $[W/(m*K)]$ \\ \hline
Beton & 20 & 0.84-1.3  \\ \hline
Drewno & - & 0.1-0.17  \\ \hline
Szkło crown & 20 & 0.22-0.29  \\ \hline
Azbest & 20 & 0.16-0.37 \\ \hline
Guma wulkanizowana & 20 & 0.22-0.29 \\ \hline
Miedź & 20 & 400 \\ \hline
Stal & 20 & 10 \\ \hline
Ołów & 20 & 30 \\ \hline
Powietrze & 20 & 0.025 \\ 
\hline
\end {tabular}
\caption{Współczynniki przewodnictwa materiałów}
\label{przewodnictwa}
\end{center}
\end {table}
\subsection{Konwekcja}
\label{Konwekcja}
Konwekcja, zwana też unoszeniem lub wnikaniem jest to zgodnie ze szkolną definicją sposób przewodnictwa ciepła polegający na
"unoszeniu pobranej energii cieplnej przez cząsteczki substancji i dzięki swojej wędrówce przekazywaniu
energii innym cząsteczkom". 
jak się znajdzie jakaś inna przystępna definicja to podmienić
Konwekcja zachodzi we wszystkich płynach, czyli zarówno cieczach jak i gazach. Nie zachodzi natomiast w ciałach stałych.
Konwekcja ze względu na połączenie w sobie dwóch zjawisk: 
\begin{itemize}
\item przekazywania ciepła
\item ruchu płynów
\end{itemize}
jest zjawiskiem niezwykle skomplikowanym do teoretycznego ujęcia. Przenoszenie ciepła w konwekcji zachodzi 
wskutek ruchu płynu, tak więc warunkiem niezbędnym do wystąpienia zjawiska konwekcji jest ruch ośrodka.
Można wyróżnić dwa podstawowe typy konwekcji, dzielące zjawisko wnikania ze względu na przyczynę ruchu ośrodka:
\begin{itemize}
\item konwekcja naturalna - w tym przypadku ruch płynu wywołany jest różnicami gęstości substancji znajdujących się w polu grawitacyjnym
\item konwekcja wymuszona - ruch płynu spowodowany jest działaniem urządzeń zewnętrznych (wentylatorów, pomp)
\end {itemize}
Przykładem konwekcji naturalnej jest unoszenie ciepłego powietrzna w pomieszczeniu. 
Ogrzane powietrze zmniejsza swoją gęstość, w wyniku czego unosi się do góry. Jego miejsce wypełnia zimne powietrze, 
które w kolejnym etapie ulega ogrzaniu rozpoczynając kolejny cykl wędrówki powietrza. Ruchy powietrza wywołane zjawiskiem 
konwekcji tworzą tzw. prądy konwekcyjne.
%TODO dodać rysunek konwekcji
 Konwekcja naturalna jest typem konwekcji występującym podczas pożaru. 

\subsection {Radiacja}
\label{Radiacja}
Radiacja, czyli inaczej promieniowanie jest to sposób rozchodzenia ciepła w postaci fal elektromagnetycznych. 
Najważniejszym aspektem przewodnictwa ciepła przez promieniowanie jest możliwość wymiany ciepła między ciałami
nie stykającymi się.
W bardzo niskich temperaturach ilość przekazywanego przy pomocy radiacji ciepła jest tak mała, że zjawisko to może być pomijane.
Wzrost znaczenia promieniowania następuje wraz ze wzrostem temperatury ciał wymieniających ciepło.
Przyjmuje się, że radiacja zachodzi dla ciał o temperaturach wyższych od $0 ^\circ$ Kelvin.
Promieniowanie jest rodzajem wymiany energii, która nie wymaga żadnego nośnika. Każde ciało emituje fale.
W normalnych warunkach większość promieniowania zachodzi przy udziale fal podczerwonych. Należy jednak pamiętać
że w radiacji mogą brać udział także fale świetlne czy ultrafioletowe. Poza emisją promieniowania każde ciało reaguje także
na fale wysyłane przez innych. Dla każdego ciała jesteśmy w stanie określić wartości trzech współczynników opisujących 
reakcję ciała na wiązkę promieniowania. Należą do nich:
\begin {itemize}
\item Absorpcyjność czyli pochłanialność
\item Refleksyjność czyli odbijalność
\item Przepuszczalność
\end {itemize}
Radiacja następuje we wszystkich kierunkach aż  do momentu zablokowania drogi promieni przez ciało pochłaniające je.
Większość ciał stałych o rozmiarach większych od kilku mikrometrów nie przepuszcza promieniowania. Na wspomnianej 
głębokości pod powierzchnią ciała następuje całkowita absorpcja promieniowania cieplnego. Ponadto ciała stałe mogą przepuszczać
fale tylko o określonej długości. Przykładem jest szkło, które przepuszcza jedynie fale świetlne.
Ilość promieniowania emitowanego przez ciała szare można obliczyć ze wzoru \ref{emisyjnosc}
\begin {equation}
\dot{Q_{emit}}=\sigma*\varepsilon*A_{s}*T_{s}^4
\label {emisyjnosc}
\end {equation}
gdzie
\begin{itemize}
\item $\varepsilon \in (0,1)$ - emisyjność powierzchni. $\varepsilon=1$ - dla ciała doskonale czarnego. Określa jak bardzo dane 
ciało jest podobne do ciała doskonale czarnego.
\item $\sigma = 5.67 * 10^-8 [W/(m^2 * K^4]) $ - stała promieniowania
\item $A_{s} [m^2]$ - powierzchnia 
\item $T_{s} [K]$ - temperatura
\end {itemize}
Przeanalizujmy przykład promieniowania między rzeczywistymi obiektami znajdującymi się w pewnym pomieszczeniu np. stół w pokoju.
W przypadku jednej powierzchni zamkniętej w innej (w omawianym przypadku wewnętrzną powierzchnią będzie powierzchnia stołu, natomiast zewnętrzną ściany pokoju) zakłada się, że wewnętrzna powierzchnia "nie opromieniowuje samej siebie".
Innymi słowy całe promieniowania ciała wewnętrznego przechodzi do powierzchni zewnętrznej. W drugim kierunku następuje tylko częściowe
przejście energii z ciała zewnętrznego do wewnątrz. Ponadto, w przypadku gdy otaczająca powierzchnia jest znacząco większa od powierzchni wewnętrznej i obie powierzchnie są oddzielone gazem, który nie promieniuje (powietrze) zjawisko promieniowania zachodzi
równolegle ze zjawiskiem konwekcji i oba te zjawiska należy wziąć pod uwagę równocześnie.
W takim przypadku wymianę ciepła można określić za pomocą wzoru \ref{emisyjnosc_konwekcja}
\begin {equation}
\dot{Q_{całk}}=\alfa_{całk}*\sigma*A_{s}*(T_{s}^4-T_\inf^4)
\label {emisyjnosc_konwekcja}
\end {equation}
gdzie:
\begin {itemize}
\item \alfa_{całk} - całkowity współczynnik wymiany ciepła
\item T_\inf - temperatura powietrza w znacznej odległości
\end {itemize} 
Omówione powyżej procesy powodują, że promieniowanie jest zjawiskiem szczególnie skomplikowanym.

\subsection {Zależność temperatury od dostarczonej energii}
Opisane w podrozdziałach \ref{Przewodnictwo}, \ref{Konwekcja}, \ref{Radiacja} metody obrazują różne sposoby przekazywania energii
między cząsteczkami materii. Po ich poznaniu należy zadać sobie pytanie w jaki sposób ta energia wpływa na temperaturę substancji?
Wielkością reprezentującą zależność między dostarczoną energią a temperaturą substancji jest ciepło właściwe. Ciepło właściwe
jest wielkością charakterystyczną dla materiału i  informuje ono o tym ile ciepła
należy dostarczyć aby ograć 1kg substancji o $1^\circ C$.Opisaną powyżej zależność przedstawia wzór \ref{cieplo_wlasciwe} 
\begin {equation}
c=Q/(m*\Delta T)
\label {cieplo_wlasciwe}  
\end {equation}
gdzie:
\begin {itemize}
\item c - ciepło właściwe $[J/ (kg * K)]$
\item m - masa ciała $[kg]$
\item T - temperatura  $[K]$
\end {itemize}
Znając ilość ciepła dostarczonego do ciała w wyniku procesów przekazywania energii oraz dokonując przekształcenia powyższego wzoru
można w bardzo prosty sposób policzyć zmianę temperatury badanego ciała.

